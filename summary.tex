% das Papierformat zuerst
\documentclass[a4paper, 11pt, fleqn]{article}
\usepackage[utf8]{inputenc}
\usepackage[T1]{fontenc}
\usepackage[margin=0.5cm]{geometry}
\usepackage[ngerman]{babel}
\usepackage{amsmath}
\usepackage{amsthm}
\usepackage{amsfonts}
\usepackage{multicol}
\usepackage{enumitem}
\usepackage{braket}
\setlist{nolistsep}

\newcommand{\setR}[0]{\mathbb{R}}
\newcommand{\setZ}[0]{\mathbb{Z}}
\newcommand{\setC}[0]{\mathbb{C}}
\newcommand{\setN}[0]{\mathbb{N}}
\newcommand{\setQ}[0]{\mathbb{Q}}
\newcommand{\setK}[0]{\mathbb{K}}
\newcommand{\setP}[0]{\mathbb{P}}
\newcommand{\spaceH}[0]{\mathcal{H}}
\newcommand{\dskal}[0]{\langle\cdot,\cdot\rangle}
\newcommand{\skal}[2]{\langle #1,#2\rangle}
\newcommand{\Ex}[1]{{\langle #1\rangle}}
\renewcommand{\labelenumi}{\theenumi}
\renewcommand{\labelenumii}{\theenumii}
\renewcommand{\vec}{\overrightarrow}
\DeclareMathOperator{\arsinh}{arsinh}
\DeclareMathOperator{\arcosh}{arcosh}
\DeclareMathOperator{\artanh}{artanh}

\newcommand{\const}{\mathrm{const.}}
\newcommand{\md}{\mathrm{d}}
\newcommand{\poisson}[5]{\sum\limits_{#5=1}^s \frac{\partial #1}{\partial #3_#5}\frac{\partial #2}{\partial #4_#5} - \frac{\partial #1}{\partial #4_#5}\frac{\partial #2}{\partial #3_#5}}
\newcommand{\pd}[2]{\frac{\partial #1}{\partial #2}}
\newcommand{\ad}[2]{\frac{\md #1}{\md #2}}
\newcommand{\ihslash}[0]{i\hslash} % TODO fancy i?
\newcommand{\fracihslash}[1]{\frac{#1}{\ihslash}}

\newcommand{\fpartial}[1]{\frac{\partial}{\partial #1}}
\newcommand{\ffpartial}[2]{\frac{\partial #1}{\partial #2}}
% hier beginnt das Dokument
\begin{document}
\shorthandoff{"}
\begin{multicols}{2}
\section{Klassische Mechanik}
Zwangsbedinungen: holonom: "Gleichungen" (holonom-rheonom zeitabhängig, holonom-skleronom nicht explizit zeitabhängig), nicht-holonom: z.~B. Ungleichungen oder diffentielle Einschränkungen.\\
Kugelkoordinaten: $z=r\cos q_1$, $x=r\sin q_1\cos q_2$, $y=r\sin q_1\sin q_2$\\
Zylinderkoordinaten: $z=h$, $x=r\cos q_1$, $y=r\sin q_1$\\
$L=T-V \qquad T\text{: kin. Energie}, V\text{: potentielle Energie}$\\
\paragraph{Lagrangegleichungen 2. Art} für holonome Zwangsbed.
\[\ad{}{t} \pd{L}{\dot q_j} - \pd{L}{q_j}=0\]
\paragraph{Zyklische Koordinate $q_j$:} $\pd{L}{q_j} = 0 \Leftrightarrow \pd{L}{\dot q_j}$ ist Erhaltungsgröße (verallgemeinerter Impuls).
\paragraph{Rezept:} Zwangsbedingungen formulieren, generalisierte Koordinaten festlegen, $L(q_i, \dot q_i, t)$ bestimmen, Lagrangegleichungen anwenden.

\paragraph{Lagrangegleichungen 1. Art} für nicht-holonome Zwangsbed., d.~h. solche, die nicht ausschl. durch die generalisierten Koordinaten ausgedrückt werden können.
\[\forall j: \quad \ad{}{t} \pd{L}{\dot q_j} - \pd{L}{q_j} = \sum\limits_i \lambda_i \underbrace{\pd{f_i}{q_j}}_{a_{ij}}\]
mit $q_j$ generalisierte Koordinaten und $f_i=0$ verbleibende Zwangsbedingungen.

\paragraph{Hamilton Prinzip}: Die Wirkung wird stationär, also $\delta S = \delta \int\limits_{t_1}^{t_2}L\md t = 0$

\paragraph{Hamilton Mechanik}: $p_i = \pd{L}{\dot q_i}$
\paragraph{Legendre Transformation} $f(x,y) \to g(u,y)$ mit $u = \pd{f}{x} $, mit $g=f-ux$, um dann die 2 verbleibenden Variablen zu bestimmen kann man $g$ total differenzieren und dann Koeffizientenvergleich mit dem "allgemeinen totalen Differential" machen.

\paragraph{Hamiltongleichungen}
\[H(q,p,t) = \sum\limits_{i=1}^n\dot q_i(q,p,t)p_i-L[q,\dot q(q,p,t),t]\]
($\dot q$ muss aufgelöst werden und darf in der finalen Gleichung nicht mehr auftauchen!)\\
das führt zu den Gleichungen:
\[\dot q_i = \pd{H}{q_i} = \{q_i, H\} \qquad \dot p_i = -\pd{H}{q_i} = \{p_i, H\}\]
\paragraph{Hamilton-Rezept}
(Wenn skleronom-holonome Zwangsbedingungen, ruhende Koordinaten, konservative Kräfte, dann $H=T+V=E$)
generalisierte Koordinaten wählen, $L = T - V$ aufstellen, $p_i = \pd{L}{\dot q_i}$ berechnen, $H=\sum\limits_{i=1}^n\dot q_i(q,p,t)p_i-L[q,\dot q(q,p,t),t]$, $\dot q_i$ ersetzen (aus dem Impuls).
\[\dot q_i = \pd{H}{q_i} \qquad \dot p_i = -\pd{H}{q_i} \qquad \text{aufstellen und zusammenfassen}\]
\[\ad{H}{t}=\pd{H}{t} =-\pd{L}{t}\]

\paragraph{Poisson-Klammern}
\[\{f,g\}_{\vec q, \vec p} = \sum\limits_{j=1}^S \left(\frac{\partial f}{\partial q_j}\frac{\partial g}{\partial p_j} - \frac{\partial f}{\partial p_j}\frac{\partial g}{\partial q_j} \right)\]

\[\ad{f}{t} = \{f,H\} + \pd{f}{t}\]
\[\{c_1f_1+c_2f_2,g\} = c_1\{f_1,g\} + c_2\{f_2,g\}, \qquad c_1,c_2=\const\]
Antisymmetrie: $\{f,g\} = -\{g,f\} \Rightarrow \{f,f\}=0$\\
Nullelement: $\{c,f\} = 0 \qquad f=f(\vec q, \vec p), c=\const$\\
Produktregel: $\{f,gh\} = g\{f,h\} + \{f,g\}h$\\
Jacobi-Identität: $\{f,\{g,h\}\} + \{g,\{h,f\}\} + \{h,\{f,g\}\} = 0$\\
$\{q_i,q_j\} = \{p_i,p_j\} = 0 \qquad \{p_i,q_j\} = \delta_{ij}$
\section{Relativitätstheorie}
$\gamma = \frac{1}{\sqrt{1 - \beta^2}}$, $\beta = \frac{v}{c}$\\
$E^2 = m^2 c^4 + \vec{p}^2 c^2 = m^2 c^2 (1 + \beta^2)$\\
Addition ($\beta$, $\beta'$): $\frac{\beta + \beta'}{1 + \beta \beta'}$

\section{Quantenmechanik}
Lichtquantum: $\omega = \frac{2 \pi c}{\lambda}$ mit $E = \hbar \omega$\\
de Broglie: $\lambda = \frac{h}{p}$\\
Unschärferelation: $\Delta p \Delta x \geq \frac{\hbar}{2}$\\
$E = h \nu = \hbar \omega, p = \hbar k$\\
$\braket{g | \alpha u + \beta v} = \alpha \braket{g | u} + \beta \braket{g | v}, \braket{g | u} = \braket{u | g}$\\
$\text{Ortsoperator:~} \hat{x} = x, \text{~~Impulsoperator:~} \hat{p} = -i \hbar \fpartial{x}$\\
$\text{Schrödingergleichung:~} i\hbar \fpartial{t} \ket{\Psi} = \underbrace{\hat{H}}_{\frac{\hat{p}}{2m}} \ket{\Psi}$\\
$\hat{H} \ket{n} = E_n \ket{n} \text{~EZ und EW von $\hat{H}$}, \text{Stationär $\rightarrow$} \ket{\Psi(t)} = e^{-\frac{i E_n t}{\hbar}}$\\
$\text{ortsunabhängige Lsg. (nach $\overset{\text{rechts}}{\underset{\text{links}}{~}}$):~} \psi = A e^{\pm i k x}$\\
$\text{vgl. bei Barriere:~} \frac{\hbar^2}{2m}\frac{\partial^2 \psi(x)}{\partial x^2} = (E - V_n) \psi(x)$\\
Lsg. der aufgeteilten Funktion via Stetigkeit von $\psi(x)$ und $\ffpartial{\psi}{x}$\\
$\langle \hat{x} \rangle = \int \psi^\ast x \psi \mathrm{d} x$, $\Delta x = \sqrt{\langle x^2 \rangle - \langle x \rangle^2}$\\
\textbf{Operatoren und so}\\
Selbstadjungiert: $A^+ = {A^\ast}^\top = A$\\
Kommutator: $[a, b] = ab - ba$\\
$[\hat{x}, \hat{p}_x] = i \hbar$, sonst $0$\\
$[\hat{A}, \hat{B} \hat{C}] = \hat{B}[\hat{A}, \hat{C}] + [\hat{A}, \hat{B}] \hat{C}$, $[\hat{A} \hat{B}, \hat{C}] = \hat{A} [\hat{B}, \hat{C}] + [\hat{A}, \hat{C}] \hat{B}$\\
\textbf{Harmonischer Oszillator} $\hat{H} = \frac{\hat{p}^2}{2m} + \frac{m \omega^2 x^2}{2}$\\
EZ von $\hat{H}$: Grundzustand $\ket{0} = N e^{-\frac{x^2}{4b^2}}, N = \sqrt[4]{2 \pi b^2}, b^2 = \frac{\hbar}{2 m \omega}$, $E = \frac{\hbar \omega}{2} = min$\\
Absteigeoperator: $a = \sqrt{\frac{m\omega}{2 \hbar}} \hat{x} + \frac{i \hat{p}}{\sqrt{2 \hbar m \omega}}$, $[a, a^+] = 1$\\
$a^+$ ist Erzeugungsoperator, $\hat{N} = a^+ a$ hermitescher Operators\\
$\hat{N} \ket{n} = n \ket{n}$ mit $\ket{n} = \frac{(a^+)^n}{\sqrt{n!}} \ket{0}$, $\hat{H} \ket{n} = \hbar \omega (n + \frac{1}{2}) \ket{n}$

\section{Mathematik}
Taylorentwicklung um $a$: $\sum_{n=0} ^ {\infty} \frac {f^{(n)}(a)}{n!} \, (x-a)^{n}$\\
$\sin(\phi) = \frac1{2i} (e^{i\phi} - e^{i\phi}), \cos(\phi) = \frac12 (e^{i\phi} + e^{-i\phi})$\\
$\sin(x \pm y)=\sin(x)\cos(y) \pm \cos(x)\sin(y)$\\
$\cos(x \pm y)=\cos(x)\cos(y) \mp \sin(x)\sin(y)$\\
$\sin (2x)= 2 \sin x \cos x$, $\cos (2x)= \cos^2 x - \sin^2 x = 1 - 2 \sin^2 x = 2 \cos^2 x - 1$, $\sin^2 x = \frac{1}{2}\ \Big(1 - \cos (2x) \Big)$\\
$e^{ix} = \cos(x) + i \sin(x)$\\
$x^2+px+q=0 \quad \Rightarrow \quad x_{1,2} = - \frac{p}{2}\pm\sqrt{\left(\frac{p}2\right)^2 - q}$
\end{multicols}
\end{document}

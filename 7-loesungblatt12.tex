\chapter{Lösungen zu Blatt 12}
Das folgende sind meine Lösungen des Blattes 12. Über Fehlerkorrekturen oder Ergänzungen freue ich mich natürlich, denn ich habe die Aufgaben nur nach bestem Wissen beantwortet.

\section{Mechanik}
\todo{$r \to \rho$}
\subsection{Aufgabe) Teilchen im Zentralpotential}
Masse $m$, $r = |\tilde{r}|$, Potential $V(r) = - \frac{\gamma}{r}$
\paragraph{a)}
Die allgemeine Langrange-Gleichung heißt hier
\begin{align*}
	L(\tilde{r}, \dot{\tilde{r}}) &= \frac12 m \dot{\tilde{r}}^2 - V(r) \overset{V(r) = - \frac{\gamma}{r}}{=} \frac12 m (\dot{x}^2 + \dot{y}^2 + \dot{z}^2) + \frac{\gamma}{r}
\end{align*}
Mit den Zylinderkoordinaten, können wir die kartesischen Koordinaten wie folgt schreiben
\begin{align*}
	x &= \rho \cos \phi\\
	y &= \rho \sin \phi\\
	z &= z
	\intertext{Und damit sind die Geschwindigkeiten}
	\dot{x} &= \dot{\rho} \cos \phi - \rho \sin \phi\\
	\dot{y} &= \dot{\rho} \sin \phi + \rho \cos \phi\\
	\dot{z} &= \dot{z}
	\intertext{Mit einfachem Einsetzen und quadrieren (oder einfach nur quadrieren der gegebenen Form von $\dot{\rho}$) findet man die Lagrangefunktion in den Zylinderkoordinaten}
	L(\rho, \dot{\rho}, \phi, \dot{\phi}, z, \dot{z}) &= \frac12 m (\dot{\rho}^2 + \rho^2 \dot{\phi}^2 + \dot{z}^2) + \frac{\gamma}{r}
	= \frac12 m (\dot{\rho}^2 + \rho^2 \dot{\phi}^2 + \dot{z}^2) + \frac{\gamma}{\sqrt{\rho^2 + z^2}}
\end{align*}

\paragraph{c)}
Damit eine Koordinate $\tau$ zyklisch ist, muss gelten
$$\ffpartial{L}{\tau} = 0$$
Womit der $\tau$-Impuls auch konstant und damit erhalten ist.\\
Wenn man sich die Lagrange-Funktion ansieht, sieht man, dass sie nicht von $\phi$ abhängt. Oder formaler
$$\ffpartial{L}{\phi} = 0$$
Damit haben wir die Erhaltungsgröße
\begin{align*}
	p_\phi &= \ffpartial{L}{\dot{\phi}} = m \rho^2 \dot{\phi} = \const
	\intertext{Und weil es hilfreich für später ist, berechnen wir auch noch $p_r$ und $p_z$}
	p_z &= \ffpartial{L}{\dot{z}} = m \dot{z} = \const\\
	p_\rho &= \ffpartial{L}{\dot \rho} = m \dot \rho
\end{align*}

\paragraph{d)}
Allgemein ist die (Euler-)Lagrange-Gleichung für einen Koordinatensatz $(\tau, \dot{\tau})$
$$\dd{t} \ffpartial{L}{\dot{\tau}} - \ffpartial{L}{\tau} = 0$$
Nun können wir für jede der 3 Koordinaten die Lagrange-Gleichung aufstellen, dabei behalten wir aber immer die Erkenntnisse aus \textbf{c)} im Hinterkopf um Rechenaufwand zu sparen.
\begin{align*}
	\dd{t} \ffpartial{L}{\dot\phi} &= 2 m \rho \dot{\rho} \dot{\phi} + m \rho^2 \ddot{\phi} &
	\Rightarrow 2 \rho \dot{\rho} \dot{\phi} + \rho^2 \ddot{\phi} &\overset!= 0\\
	\dd{t} \ffpartial{L}{\dot z} &= m \ddot{z} & &
	\\
	\ffpartial{L}{z} &= - \frac{\gamma z}{\sqrt{\rho^2 + z^2}^3} &
	\Rightarrow m \ddot{z}  + \frac{\gamma z}{\sqrt{\rho^2 + z^2}^3}&\overset!= 0\\
	\dd{t} \ffpartial{L}{\dot{\rho}} &= m \ddot{\rho} && \\
	\ffpartial{L}{\rho} &= m \rho \dot{\phi}^2 - \frac{\gamma \rho}{\sqrt{\rho^2+z^2}^3} &
	\Rightarrow m\ddot{\rho} - m \rho \dot{\phi}^2 + \frac{\gamma \rho}{\sqrt{\rho^2+z^2}^3} &\overset!= 0
\end{align*}
\paragraph{e)}
Annahmen: $z=0$, $\dot{\rho} = 0$ und $\ddot{\phi} = 0$\\
Aus der Annahme folgt offensichtlich auch $\dot{z} = 0, \ddot{z} = 0, \ddot{\rho} = 0$, $\rho = \const$ und $\dot{\phi} = \const$
\begin{align*}
	 m\ddot{\rho} - m \rho \dot{\phi}^2 + \frac{\gamma \rho}{\sqrt{\rho^2+z^2}^3} &\overset!= 0 \\
	\xLeftrightarrow{\text{mit Annahme}} \frac{\gamma \rho}{\rho^3} &\overset!= m \rho \dot{\phi}^2 \\
	\Leftrightarrow \gamma &\overset!= m \rho^3 \dot{\phi}^2 \\
	\Leftrightarrow m \rho \gamma &\overset!= m^2 \rho^4 \dot{\phi}^2
\end{align*}

\paragraph{f)}
Erinnerung: $p_\phi = m \rho^2 \dot{\phi} = l$
\begin{align*}
	\rho(l) &= \rho(m \rho^2 \dot\phi) = \frac{l^2}{\gamma m}
\end{align*}
Wenn wir jetzt $l' = 2 l$ einsetzen
\begin{align*}
	\rho' &= \rho(l') = \frac{4 l^2}{\gamma m} = 4 \rho(l)
\end{align*}
\paragraph{g)}
\begin{align*}
	H &= \underbrace{m \dot{\rho}^2} + m \rho^2\dot{\phi}^2 + m \dot{z}^2 - \frac12 m (\dot{\rho}^2 + \rho^2 \dot{\phi}^2 + \dot{z}^2) - \frac{\gamma}{\sqrt{\rho^2+z^2}}\\
	&= \frac12 m \dot{\rho}^2 + m \dot{\phi}^2 (\rho^2 - \frac{\rho^2}{2}) + \frac12 m \dot{z}^2 - \frac{\gamma}{\sqrt{\rho^2+z^2}}\\
	&= \frac12 \frac{p_\rho^2}{m} + \frac{p_\phi^2}{2 m\rho^2} + \frac12 \frac{p_z^2}{m} - \frac{\gamma}{\sqrt{\rho^2+z^2}}
\end{align*}

\paragraph{i)}
\begin{align*}
	\{p_z, H\} &= \frac1{2m} \underbrace{\{p_z, p_\rho^2\}}_{=0} + \underbrace{\{p_z, \frac{p_\phi^2}{2 m\rho^2}\}}_{=0} + \frac1{2m} \underbrace{\{p_z, p_z^2\}}_{=0} - \underbrace{\{p_z, \frac{\gamma}{\rho}\}}_{=0}\\
	&= 0
\end{align*}


\subsection{Aufgabe) Lagrangegleichungen \& Erhaltungssätze}
\paragraph{a)}
\begin{align*}
	V &= - (mz_1 + Mz_2) g\\
	T &= \frac12 (m \dot{z_1}^2 + M \dot{z_2}^2) 
\end{align*}
\paragraph{b)}
\begin{align*}
	L &= T - V = \frac12 (m \dot{z_1}^2 + M \dot{z_2}^2) + (mz_1 + Mz_2) g\\
	\intertext{mit $l = \pi R + z_1 + z_2 ~\Rightarrow~ z_2 = l - z_1 - \pi R$ und $\dot{z_2} = - \dot{z_1}$}
	L &= \frac12 (m + M) \dot{z_1}^2 + mg z_1 + Mg (l - z_1 - \pi R)
	\intertext{da $R\ll l$}
	  &\approx \frac12 (m + M) \dot{z_1}^2 + mg z_1 + Mg (l - z_1)	
\end{align*}

\paragraph{c)}
%NEIN, Impulserhaltung gilt nicht
%$$p_m + p_M = 0 \qquad \Rightarrow \qquad m z_1 + M z_2 = \const$$
Energieerhaltung?
\paragraph{d)}
\begin{align*}
	\dd{t} \ffpartial{L}{\dot{z}} &= (m + M) \ddot{z_1}\\
	\ffpartial{L}{z_1} &= (m - M) g\\
	\Rightarrow  (m + M) \ddot{z_1} - (m - M) g &\overset!= 0\\
	\Rightarrow \ddot{z_1} = \frac{m - M}{m + M} g
\end{align*}
Da laut Aufgabenstellung gilt $m < M$, gilt, dass die Masse $m$ sich beschleunigt von der Erde nach oben entfernt, dass also die Masse $M$ immer schneller zum Boden hin fällt. Die Beschleunigung der Masse ist mit $z_1 = l - \pi R - z_2$ und damit $z_2 \sim -z_1$:
$$\ddot{z_2} = -\frac{m - M}{m + M} g$$

\paragraph{e)}
$$z_1(t) = z_0 + v_0 t + \frac12 \frac{m - M}{m + M} g t^2$$
Damit bewegt sich die Masse $m$ beschleunigt nach oben.

\section{Spezielle Relativitätstheorie}
\subsection{Relativistisches Dinosaurierei}
$v = 0.9c$. $t' = \SI{30}{\year}$ (im Bezugssystems der Rakete).

\paragraph{a)}
"`Bewegte Uhren gehen langsamer"'
$$t = \gamma t' = \frac{1}{\sqrt{1 - \frac{9^2}{10^2}}} t' = \frac{10}{\sqrt{19}} \cdot \SI{30}{\year} \approx \SI{68.82}{\year}$$

$$s = vt \approx \SI{61,94}{\lightyear}$$

\paragraph{b)}
$$l = \frac{l'}{\gamma} = \frac{\sqrt{19}}{10} \cdot \SI{15}{\cm} \approx  \SI{6.54}{\cm}$$

\paragraph{c)}
$$t = \gamma t' ~ \Rightarrow ~ \frac{t}{t'} = \gamma ~ \Rightarrow ~ v = \sqrt{1 - 10^{-12}} \cdot c$$

\subsection{Geschwindigkeitsaddition}
Einfaches Ausrechnen, wie bei einem der Übungsblätter.

\section{Quantenmechanik}

\subsection{Aufgabe) Zerfließendes Wellenpaket}
$m = \SI{1e-3}{\kilogram}$, $t = 0 \leadsto \Delta x = \SI{e-10}{\meter}$
Im folgenden gehen wir einfach wie im Hinweis skizziert vor.
\begin{align*}
	\Delta x(t) &= \sqrt{a (1 + (\Delta t)^2} = \sqrt{a (1 + (\frac{\hbar t}{2ma})^2)}
	\intertext{Zuerst berechnen wir $a$}
	\Delta x(t) &= \sqrt{a} \overset{!}{=} \SI{e-10}{\meter}\\
	\Rightarrow a &= \SI{e-20}{\meter}
	\intertext{Nun formen wir das Ganze nach $t$ um undc erhalten}
	t &= 2 m a \frac1\hbar \sqrt{\frac{(\Delta x(t))^2}{a} - 1}
	\intertext{Sei nun (wie im Hinweis) $\Delta x = \SI{2e-10}{\meter}$, dann gilt}
	t &= 2 m a \frac1\hbar \sqrt{\frac{(\SI{2e-10}{\meter})^2}{\SI{e-20}{\meter}} - 1} = 2 m a \frac1\hbar \sqrt{4 - 1} = \frac{3ma}{\hbar}
	\intertext{Ein $\alpha$-Teilchen braucht für das Zerfließen des Wellenpakets offensichtlich viel kürzer, da $m = \SI{e-3}{\kilogram} \ll m_\alpha = \SI{4e-27}{\kilogram}$}
\end{align*}

\subsection{Aufgabe) Rechteckpotential}
Rechteckpotential, um $x_0$ zentriert, mit Breite $a$ und Höhe $V_0$. Teilchen hat Masse $m$.

\paragraph{a)}
$0 < E < V_0$. Teilchen nähert sich der Barriere von links.\\
Schwingung links auf mittlerem Niveau \conseq $e^{-x}$ artige Kurve im Potential \conseq Schwingung mit anfänglicher Frequenz auf niedrigerem Niveau.

\paragraph{b)}
Allgemein ist die stationäre Schrödingergleichung
\begin{align*}
	\frac{\hbar^2}{2m}\frac{\partial^2 \psi(x)}{\partial x^2} +  V_n \psi(x) &= E \psi(x)
	\intertext{diese kann man Umschreiben zu}
	\frac{\partial^2 \psi(x)}{\partial x^2} &= \underbrace{\frac{2m}{\hbar^2} (E - V_n)}_{= k^2} \psi(x)
	\intertext{Da $E - V_0 < 0$, ziehen wir das Minus aus der Wurzel}
	k &= i \underbrace{\sqrt{2m \frac{V_0 - E}{\hbar^2}}}_{\kappa}\\
	\intertext{Nun setzen wir dies in den Ansatz ein}
	e^{ikx} = e^{-\kappa x} = e^{- \sqrt{2m \frac{V_0 - E}{\hbar^2}} x}
\end{align*}
Das $\kappa \propto \sqrt{V}$, $\kappa \propto -\sqrt{E}$,  $\kappa \propto \sqrt{m}$ und $\kappa$ in keiner Beziehung zu $a$ steht, sollte offensichtlich sein.

\paragraph{c)}
Ansatz
\begin{align*}
	\psi &= \begin{cases}
		A e^{ik_1x} + Be^{-ik_1x} & x < -\frac{a}{2}\\
		C e^{ik_2 x} + De^{-ik_2 x} & -\frac{a}{2} \leq x \leq \frac{a}{2}\\
		E e^{ik_1x} + F e^{-ik_1x} & x > \frac{a}{2}
	\end{cases}
	\intertext{Wie im Aufgabenteil \textit{b)} finden wir $k_1$ und $k_2$. Nur das diesmal $E - V_0 > 0$ und damit kein Vorzeichen herausgezogen werden muss.}
	k_1 &= \sqrt{2m \frac{E - \underbrace{V}_{= 0}}{\hbar^2}}\\
	k_2 &= \sqrt{2m \frac{E - V_0}{\hbar^2}}
	\intertext{Da $A e^{ikx} + Be^{-ikx} \propto \sin(k x)$ können wir folgern, dass bei größeren $k$ die Wellenlänge kleiner ist. Offensichtlich ist $k_2 < k_1$, somit ist auch die Wellenlänge im mittleren Bereich größer und in den beiden äußeren Bereichen kleiner.}
\end{align*}

\subsection{Aufgabe) Rechteckpotential}
Ein Potentialtopf um $0$ zentriert, mit der Breite $a$ und dem Potential $V_0 < E < 0$.

\paragraph{a)}
Im klassisch verbotenen Bereich, ist es eine jeweils abfallende $e$-Funktion. Im mittleren Bereich zeichnet man eine um $0$ zentrierte $\cos$-Funktion mit der Wellenlänge $a$. Hierbei zu beachten ist, dass $E < 0$.
$$\begin{cases}
	A e^{\kappa x} = A e^{- i^2 \sqrt{\frac{2m}{\hbar^2} (-E)}} & x < -\frac{a}{2}\\
	A e^{-\kappa x} = A e^{- (- i^2) \sqrt{\frac{2m}{\hbar^2} (-E)}} & x > \frac{a}{2}\\
	B e^{i k x} + C e^{-ikx} & \text{sonst}
\end{cases}$$
Wobei $k = \sqrt{\frac{2m}{\hbar^2} \underbrace{(E - V_0)}_{> 0}}$.

\paragraph{c)}
$E > 0$. Wir erinnern uns an die vorherige Aufgabe. Offensichtlich hat die Wellenfunktion folgende Struktur:
\begin{align*}
\psi &= \begin{cases}
A e^{ik_1x} + Be^{-ik_1x} & x < -\frac{a}{2}\\
C e^{ik_2 x} + De^{-ik_2 x} & -\frac{a}{2} \leq x \leq \frac{a}{2}\\
E e^{ik_1x} + F e^{-ik_1x} & x > \frac{a}{2}
\end{cases}
\intertext{$k_1$ und $k_2$ findet man wie gewohnt}
k_1 &= \sqrt{2m \frac{E - \underbrace{V}_{= 0}}{\hbar^2}}\\
k_2 &= \sqrt{2m \frac{E - \underbrace{V}_{< 0}}{\hbar^2}}\\
\Rightarrow k_1 < k_2
\intertext{Damit gilt mit der Begründung von der letzten Aufgabe, dass die Wellenlänge im mittleren Bereich kleiner, als jene in den äußeren Bereichen ist.}
\end{align*} 

\subsection{Aufgabe) Hilbertraumvektoren}
$$ \hat{S}_x = \frac{\hbar}{2} \begin{pmatrix}
0 & 1\\ 1 & 0
\end{pmatrix}$$
\paragraph{a)}
Zuerst berechnen wir den Normierungsvektor $N$ über die Wahrscheinlichkeitsbetrachtung der Eigenvektoren, für die gelten muss $\braket{x | x} = 1$:
\begin{align*}
	\braket{U_x | U_x} = N^2 \binom{1}{1}^\top \binom{1}{1} = N^2 \cdot 2 \qquad \Rightarrow \qquad N = \frac{1}{\sqrt{2}}
\end{align*}
Nun berechnen wir noch die beiden Eigenwerte. Für diese gilt nach Definition $\hat{S}_x \ket{x} = \lambda \ket{x}$. Konkret gilt
\begin{align*}
	\hat{S}_x \ket{U_x} &= \frac{\hbar}{2} N \binom{1}{1} \overset{!}{=} \lambda_{U_x} N \ket{U_x}\\
	\Rightarrow \lambda_{U_x} = \frac{\hbar}{2}\\
	\hat{S}_x \ket{D_x} &= \frac{\hbar}{2} N \binom{-1}{1} \overset{!}{=} \lambda_{D_x} N \binom{1}{-1} = \lambda_{D_x} N \ket{U_x}\\
	\Rightarrow \lambda_{U_x} = -\frac{\hbar}{2}
\end{align*}

\paragraph{b)} Betrachteter Vektor: $\ket{\psi} = N_\psi \binom{1 + \sqrt{3}}{1- \sqrt{3}}$
\begin{align*}
	\text{Normierung: ~} \braket{\psi | \psi} = N_\psi^2 ((1 + \sqrt{3})^3 + (1 - \sqrt{3})^2) = N_\psi^2 \cdot 8 \qquad \Rightarrow \qquad N_\psi = \frac{1}{\sqrt{8}}
\end{align*}
Man sieht auf den ersten Blick, dass $\ket{\psi}$ auch wie folgt geschrieben werden kann (Achtung mit den Normierungsfaktoren):
$$\underbrace{\frac{\sqrt{2}}{\sqrt{8}}}_{= \frac{1}{2}} (\ket{U_x} + \sqrt{3} \ket{D_x})$$

\paragraph{c)}
Da schon alles wichtige in der Aufgabenstellung angegeben ist, kann man einfach ausrechnen und findet dann
$$P_U = \frac{1}{4} \qquad P_D = \frac34$$
Anmerkung: Umbedingt immer die Wahrscheinlichkeiten addieren, addieren sich diese nicht zu 1 auf, ist was falsch.

\paragraph{d)}
Das zu zeigen ist durch einfaches Ausrechnen möglich:
\begin{align*}
	\hat{S}_x^2 = \frac{\hbar}{2} \begin{pmatrix}
	0 & 1\\ 1 & 0
	\end{pmatrix} \cdot \frac{\hbar}{2} \begin{pmatrix}
	0 & 1\\ 1 & 0
	\end{pmatrix} = \frac{\hbar^2}{4} \begin{pmatrix}
	1 & 0\\ 0 & 1
	\end{pmatrix} 
\end{align*}

%%% Local Variables:
%%% mode: latex
%%% TeX-master: "document"
%%% End:

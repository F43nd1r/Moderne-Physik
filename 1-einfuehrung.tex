\chapter{Einführung}

\begin{definition*}[Moderne Physik]
	Die moderne Physik steht im Gegensatz zur "`Klassischen Physik"', die bis Anfang des ersten Viertel des 20. Jahrhunderts vorherrschend war. Die klassische Physik besteht im Wesentlichen aus der Newtonschen Mechanik und der Maxwellschen Elektrodynamik.
	Das vorherrschende Paradigma in diesem Zweig der Physik ist und war, dass alles im Prinzip berechenbar ist, solange man die Anfangsbedingungen kennt und damit auch die zeitliche Entwicklung eines Systems vorhersagen kann.
\end{definition*}

\paragraph{Aber:} Experimente zeigten im Laufe der Zeit immer mehr Widersprüche zur klassischen Physik. Im Folgenden werden ein paar von ihnen angegeben:
\begin{description}
	\item[Michelson-Morley] Es wurde gezeigt, dass es keinen "`Äther"' gibt, durch den sich das Licht bewegt und dass die Lichtgeschwindigkeit konstant ist.
	\item[\conseq] \textbf{Spezielle Relativitätstheorie} auf die in einem späteren Kapitel noch eingegangen wird.
	\item[Diskrete Emissionsspektren (Spektrallinien)] sind bei Strahlung emittierenden Objekten messbar.
	\item[Welleneigenschaft von Teilchen] vgl. Spaltexperimente mit Elektronen \footnote{Aufbau: Elektronen werden auf einen Doppelspalt "`geschossen"'. Dahinter befindet sich in einiger Entfernung ein Detektor. Klassisch würde man erwarten, dass ein Elektron ein Teilchen ist und damit der Detektor nur auf zwei schmalen Streifen Elektronen detektiert. Im Experiment detektiert man dagegen ein Interferenzmuster, das an jenes von Wellen erinnert. Vgl. \href{http://de.wikipedia.org/wiki/Doppelspaltexperiment}{Wikipedia}}, es entsteht ein Widerspruch zur klassischen Physik, denn es sind nur die Wahrscheinlichkeiten vorhersagbar mit der sich ein Elektron zu einem bestimmten Zeitpunkt an einem bestimmten Ort befindet.
	\item[Teilcheneigenschaften von Lichtwellen] vgl. \href{http://de.wikipedia.org/wiki/Photoelektrischer_Effekt}{Photoelektrischer-Effekt}
	\item[Schwarzkörperspektrum] Die Abhängigkeit des emittierten Lichtspektrums eines Körpers oder Gases von der Temperatur. Das (rein gedankliche) Schwarzkörperspektrum widerspricht der Boltzmann-Verteilung. Daraus folgerte Planck, dass die untersuchten Teilchen (des Gases oder Körpers) ununterscheidbar oder identisch sind.
	\item[\conseq] \textbf{Quantenphysik} auf die in einem späteren Kapitel noch eingegangen wird.
\end{description}
Das nächste Kapitel behandelt die klassische Mechanik (ein Teilgebiet der klassischen Physik), da diese notwendig zum Verständnis der modernen Physik ist.

%%% Local Variables:
%%% mode: latex
%%% TeX-master: "document"
%%% End:

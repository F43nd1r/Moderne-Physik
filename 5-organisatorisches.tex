\chapter{Organisatorisches}

\section{Literatur}

\begin{itemize}
	\item Teubner-Taschenbuch der Mathematik \footnote{ehemals Bronstein, Semendjajew, \dots} Teubner Verlag \conseq \textit{Gute und zusammenfassende Formelsammlung und Integraltabellen, gut auf dem Schreibtisch zu haben}
	\item S. Grossmann, Mathematischer Einführungskurs in die Physik, Teubner Verlag \conseq \textit{Die Wichtigsten Hilfsmittel für die theoretische Physik}
	\item Schäfer/Georgi/Trippler, Mathematik-Vorkurs, Teubner Verlag \conseq \textit{Abitur-Stoff und etwas mehr}
	\item L. Papula, Mathematik für Ingenieure und Naturwissenschaftler, Vieweg Verlag
	\item P. Furlan, Das gelbe Rechenbuch, Verlag Martina Furlan
	\item F. Kuypers, Klassische Mechanik, 5. Auflage, Wiley-VCH
	\item I. Honerkamp, H. Römer, Klassische theoretische Physik, \href{http://www.freidok.uni-freiburg.de/volltexte/82/}{digitalisierte 3. Auflage}
	\item F. Hund, Grundbegriffe der Physik, BI Hochschulbücher (sehr alt), gibt's in der Fachbibliothek
\end{itemize}

\subsection{Gute Lernmaterialen}
Eine Liste von Materialien, welche beim Lernen helfen und welche besser sind, als die wenigen Altklausuren\footnote{Die Fachschaft hat nur 4 Altklausuren ohne Musterlösungen}.
\begin{description}
	\item[Grundkurs Theoretische Physik 2, Nolting] Ein gutes Buch mit vielen Aufgaben zu Langrange und Hamilton. Es ist in der Bibliothek in ausreichender Menge vorhanden.
	\item[\href{https://de.wikibooks.org/wiki/Aufgabensammlung_Physik:_Theoretische_Mechanik}{wikibooks: Aufgaben Theorische Mechanik}] Einfache Aufgaben zum Langrange (wobei das spherische Pendel in der letzten Klausur dran kam), welche gut und ausführlich erklärt sind.
	\item[Blätter des letzten Jahres] Im letzten Jahr, waren 7 von 13 Übungsblättern zum Thema Quantenmechanik. Sie waren einfacher, also näher an Klausuraufgaben, und verständlicher als die aktuellen. Sie werden leider nicht mehr auf der Website der Vorlesung verlinkt, sind aber dennoch zugänglich. Die Blätter \verb|B| und Lösungen \verb|L| findet ihr unter\\ \verb-http://www.tkm.kit.edu/downloads/ss2014_modphys_info/mod_phys_- \verb-inf_2014_{B|L}[Nummer].pdf-
	\item[\href{http://physik.leech.it/pub/}{http://physik.leech.it/pub/}] Sammlung von Altklausuren von Physikstudenten. Die Klausuren zu Theo B (Analytische Mechanik) und Theo D (Quantenmechanik) sind teilweise zur Vorbereitung geeignet. Denn meine Erfahrung ist, für dieses Fach, dass es besser ist den voll und ganz zu verstehen, als dutzende von Altklausuren zu rechnen.
	\item[Quantenphysik für Dummies / Steven Holzner] Ein Buch, dass die Quantenmechanik gut erklärt. Es ist in der Bib in der Lehrbuchsammlung Physik zu finden.
\end{description}

%%% Local Variables:
%%% mode: latex
%%% TeX-master: "document"
%%% End:
